\chapter{Введение}

Копирование при записи (Copy-On-Write, далее COW) это альтернатива
журналированию (Journaling, Write-Ahead-Logging, далее WAL). При использовании
WAL изменения перед тем как быть зафиксированы выписываются в специальное место
 - журнал. Только после того как все необходимые данные зафиксированы в
журнале, журнал будет проигран, т. е. изменения будут зафиксированы в файловой
системе. При таком подходе данные приходится выписывать на диск дважды: первый
раз при записи журнала, второй раз при проигрывании журнала, поэтому некоторые
файловые системы по-умолчанию журналируют только метаданные, но не делают этого
для данных.

В отличие от WAL, при использовании COW файловая система не перезаписывает
данные на месте, вместо этого алоцируется новое место на диске, данные
записываются в новое место после чего подменяется ссылка на данные так, чтобы
она указывала на новое место. Среди файловых систем использующих COW такие
файловые системы как WAFL~\cite{WAFL}, ZFS и Btrfs.

В этой работе рассматривается дизайн и реализация COW файловой системы
использующей в качестве основной индексной структуры данных
журнально-структурированное дерево со слиянием (Log-Strucutred Merge Tree, далее
LSM-дерево) под названием myfs~\cite{MYFS}.

Работа состоит из трех разделов. В первом разделе будут рассмотрены некоторые
достоинства использования COW вместо WAL, а также файловые системы ZFS и Btrfs
применяющие этот подход\footnote{WAFL является закрытой разработкой компании
NetApp, поэтому о ее внутреннем устройстве и современном состоянии можно судить
только по статьям, в связи с этим рассматриваться в данной работе она не
будет.}. Во втором разделе рассматривается дизайн и реализация myfs, а также
темы непосредственно связанные с реализацией. Третий и заключительный раздел
посвящен результатам тестирования и выводам по проделанной работе.

\section{Копирование при записи}

Как упоминалось ранее, при использовании COW при записи новых данных или
изменении старых файловая система алоцирует новое место на диске и создает
новую версию. Одним из достоинств использования такого подхода является
возможность эффективной реализации снимков состояния (snapshot), в то время как
файловым системам не использующим COW такая функциональность не доступна.

\subsection{Снимки состояния}

При использовании COW реализация снимков состояния является в некоторой степени
простой. Так как при изменении COW создает новую копию данных, все что нужно для
реализации снимков - это сохранить ссылку на копию данных и запретить
переиспользование места, занятого этой копией данных. Если смотреть на всю
файловую систему целиком как на одно единственное дерево, то при обновлении
необходимо сначала найти нужный узел дерева, затем создать копию этого узла и
изменить ее соответствующим образом, после чего необходимо обновить ссылку на
этот узел в родительском узле. Обновление ссылки в родительском узле так же
является изменением, соответственно, требуется опять же создать копию и изменить
копию. Процесс копирования заканчивается, когда файловая система создаст новый
корень дерева. Таким образом старый и новый корень дерева будут представлять
версии файловой системы до и после изменения соответственно, а чтобы создать
снимок состояния файловой системе достаточно сохранить текущий корень.

Естественно возможность создания снимков состояния не была бы важной если бы
снимкам не было практического применения. Снимки состояния позволяют получить
доступ к консистентному стабильному состоянию файловой системы. Эту возможность
можно использовать, чтобы сохранить состояние файловой системы перед какими-либо
изменениями, например, такими как установка или удаление приложений. Если по
каким-то причинам потребуется откатить изменения, то для этого можно
использовать созданный снимок.

Другое применения снимков состояния - это резервное копирование. Создать
резервную копию данных, когда данные постоянно изменяются довольно
проблематично. Поэтому снимки состояния, так как они предоставляют доступ к
созраненному состоянию файловой системы позволяют упростить задачу. Все что
требуется это создать снимок и создать резервную копию из этого снимка.

\subsection{Транзакции}

Другое достоинство использования COW - это обеспечение транзакционности. Ссылка
на обновленную версию данных или новые данные обновляется когда данные уже
записаны на диск, таким образом на диске не может быть записана только часть
транзакции. Транзакция либо применяется полностью, когда файловая система
создает новый корень, либо не применяется вовсе.

\section{ZFS}

ZFS сравнительно молодая файловая система\footnote{ZFS была анонсирована в 2005
году}, разработанная в компании Sun Microsistems под руководством Джефа Бонвика.
В отличие от традиционных файловых систем ZFS сочетает в себя функциональность
файловой системы вместе с управлением разделами. Или более точно, в ZFS вместо
разделов используются пулы дискового пространства и по мере необходимости можно
добавлять или удалять диски из пула. Таким образом, ZFS может работать с
несколькими дисками одновременно и организовывать избыточное хранение данных и
метаданных не полагаясь на аппаратные или программные дисковые массивы.

ZFS во многих смыслах является очень инновационной файловой системой, но
рассмотрение множества ее возможностей выходит за пределы данной работы. Данная
работа ограничивается поверхностным рассмотрением дискового формата файловой
системы в объеме достаточном для понимания того, как работает COW в ZFS и какими
индексными структурами данных она пользуется. Детальное описание дискового
формата может быть найдено в исходных кодах, а также в документации
~\cite{ZFSSpec}.

Условно файловую систему ZFS можно разделить на несколько уровней. Самым базовым
уровнем является DSU. На этом уровне, собственно, организовано хранение данных в
виде объектов и наборов объектов, так же на этом уровне работает COW. Поверх
интерфейса DSU работают другие уровни, например, уровень ZAP хранит внутри
объектов пары имя/значение и таким образом отвечает за индексацию, а ZPL
использует DSU и ZAP чтобы предоставить интерфейс POSIX совместимой файловой
системы.

\subsection{Объекты и группы объектов в ZFS}

За непосредственное представление данных на диске в ZFS отвечает Data Management
Unit (далее DSU). DSU позволяет хранить объекты и группы объектов, поверх
объектного интерфейса ZFS реализует интерфейс POSIX совместимой файловой
системы.

В качестве указателя на блок данных используются структура blkptr\_t:
\begin{lstlisting}
typedef struct dva {
    uint64_t     dva_word[2];
} dva_t;

typedef struct blkptr {
    dva_t        blk_dva[SPA_DVAS_PER_BP]; /* Data Virtual Addresses */
    uint64_t     blk_prop;       /* size, compression, type, etc     */
    uint64_t     blk_pad[2];     /* Extra space for the future       */
    uint64_t     blk_phys_birth; /* txg when block was allocated     */
    uint64_t     blk_birth;      /* transaction group at birth       */
    uint64_t     blk_fill;       /* fill count                       */
    zio_cksum_t  blk_cksum;      /* 256-bit checksum                 */
} blkptr_t;
\end{lstlisting}

Непосредственно на данные указывает поле blk\_dva и в blkptr\_t может хранится
до трех таких указателей на разные позиции. Таким образом одна и та же порция
данных может быть продублирована в нескольких местах на диске.

Каждый указатель в массиве blk\_dva представляет 128 битное значение, которое
кодирует номер устройства и номер 512 байтового блока внутри этого устройства.

blkptr\_t может указывать на блок, содержащий другие структуры blkptr\_t. В
терминологии ZFS такие блоки называются групповыми блоками (gang blocks). А
указатели таким образом могут образовывать дерево, листья которого ссылаются на
данные.

Для представления объектов в ZFS используется структура dnode\_phys\_t:
\begin{lstlisting}
typedef struct dnode_phys {
    uint8_t  dn_type;         /* dmu_object_type_t */
    uint8_t  dn_indblkshift;  /* ln2(indirect block size) */
    uint8_t  dn_nlevels;      /* 1=dn_blkptr->data blocks */
    uint8_t  dn_nblkptr;      /* length of dn_blkptr */
    uint8_t  dn_bonustype;    /* type of data in bonus buffer */
    uint8_t  dn_checksum;     /* ZIO_CHECKSUM type */
    uint8_t  dn_compress;     /* ZIO_COMPRESS type */
    uint8_t  dn_flags;        /* DNODE_FLAG_* */
    uint16_t dn_datablkszsec; /* data block size in 512b sectors */
    uint16_t dn_bonuslen;     /* length of dn_bonus */
    uint8_t  dn_extra_slots;  /* # of subsequent slots consumed */
    uint8_t  dn_pad2[3];

    uint64_t dn_maxblkid;     /* largest allocated block ID */
    uint64_t dn_used;         /* bytes (or sectors) of disk space */

    uint64_t dn_pad3[4];

    /*
     * The tail region is 448 bytes for a 512 byte dnode, and
     * correspondingly larger for larger dnode sizes. The spill
     * block pointer, when present, is always at the end of the tail
     * region. There are three ways this space may be used, using
     * a 512 byte dnode for this diagram:
     *
     * 0       64      128     192     256     320     384     448 (offset)
     * +---------------+---------------+---------------+-------+
     * | dn_blkptr[0]  | dn_blkptr[1]  | dn_blkptr[2]  | /     |
     * +---------------+---------------+---------------+-------+
     * | dn_blkptr[0]  | dn_bonus[0..319]                      |
     * +---------------+-----------------------+---------------+
     * | dn_blkptr[0]  | dn_bonus[0..191]      | dn_spill      |
     * +---------------+-----------------------+---------------+
     */
    union {
        blkptr_t     dn_blkptr[1+DN_OLD_MAX_BONUSLEN/sizeof (blkptr_t)];
        struct {
            blkptr_t __dn_ignore1;
            uint8_t  dn_bonus[DN_OLD_MAX_BONUSLEN];
        };
        struct {
            blkptr_t __dn_ignore2;
            uint8_t  __dn_ignore3[DN_OLD_MAX_BONUSLEN - sizeof (blkptr_t)];
            blkptr_t dn_spill;
        };
    };
} dnode_phys_t;
\end{lstlisting}

На каждый объект можно смотреть как на разреженный файл, а поле dn\_blkptr
хранит корни деревьев описывающих расположение содержимого файла на дисках.
При изменении содержимого объекта, алоцируется новое место на диске, в это
новое место записывается измененная порция объекта, после чего обновляются все
указатели вплоть до корня, хранящегося внутри структуры dnode\_phys\_t.

Объекты можно объединять в группы. Группа объектов сама представляется как
объект. Содержимое объекта хранится как массив dnode\_phys\_t, где позиция в
массиве уникально идентифицирует объект.

\subsection{Индексация}

За индексацию в ZFS отвечает специальный модуль ZFS Attribute Processor (ZAP).
Модуль ZAP оперирует специальными типами объектов, который хранят пары
имя/значение. Имя может быть любой строкой символов завершающейся нулевым
символом длинной до 256 байт включая завершающий нулевой символ. В качестве
значения выступает массив целых чисел.

ZFS поддерживает две различные вариации внутренней структуры ZAP объектов:
microzap и fatzap. Выбор между двумя вариациями зависит от следующих условий:

\begin{itemize}
  \item помещаются ли все пары имя/значение в один блок\footnote{Максимальный
        размер блока в ZFS ограничен 128Kb.};
  \item помещается ли значение всех пар в uint64\_t;
  \item помещается ли имя всех пар в 50 байт включая нулевой;
\end{itemize}

Если все три условия выполняются, то используется microzap, в противном случае
используется fatzap. Другими словами microzap позволяет эффективно работать с
небольшими наборами пар, в то время как fatzap используется для больших наборов.

\subsubsection{Microzap}

Microzap объект состоит из одного единственного непрерывного блока данных,
который содержит массив записей следующего формата:
\begin{lstlisting}
typedef struct mzap_ent_phys {
    uint64_t mze_value;
    uint32_t mze_cd;
    uint16_t mze_pad;
    char     mze_name[MZAP_NAME_LEN];
} mzap_ent_phys_t;
\end{lstlisting}

Этот массив представляет из себя ничто иное как хештаблицу с открытой
адресацией.

\subsubsection{Fatzap}

Как и microzap fatzap представляет из себя хештаблицу использующую часть бит
64-битного хеша имени пары, но в отличие от microzap использует не открытую
адресацию, а цепочки для разрешения коллизий.

В заголовке fatzap объекта содержится следующая структура, которая хранит
информацию о ZAP объекте и ссылку на таблицу указателей (хеш-таблицу).
\begin{lstlisting}
typedef struct zap_phys {
    uint64_t zap_block_type; /* ZBT_HEADER */
    uint64_t zap_magic;      /* ZAP_MAGIC */

    struct zap_table_phys {
        uint64_t zt_blk;         /* starting block number */
        uint64_t zt_numblks;     /* number of blocks */
        uint64_t zt_shift;       /* bits to index it */
        uint64_t zt_nextblk;     /* next (larger) copy start block */
        uint64_t zt_blks_copied; /* number source blocks copied */
    } zap_ptrtbl;

    uint64_t zap_freeblk;     /* the next free block */
    uint64_t zap_num_leafs;   /* number of leafs */
    uint64_t zap_num_entries; /* number of entries */
    uint64_t zap_salt;        /* salt to stir into hash function */
    uint64_t zap_normflags;   /* flags for u8_textprep_str() */
    uint64_t zap_flags;       /* zap_flags_t */
} zap_phys_t;
\end{lstlisting}

Таблица указателей хранит указание на место внутри объекта, где хранится первая
запись цепочки. Вся записи связаны в односвязный список и каждая запись имеет
следующий формат:

\begin{lstlisting}
typedef struct zap_leaf_phys {
    struct zap_leaf_header {
        uint64_t lh_block_type;  /* ZBT_LEAF */
        uint64_t lh_pad1;
        uint64_t lh_prefix;      /* hash prefix of this leaf */
        uint32_t lh_magic;       /* ZAP_LEAF_MAGIC */
        uint16_t lh_nfree;       /* number free chunks */
        uint16_t lh_nentries;    /* number of entries */
        uint16_t lh_prefix_len;  /* num bits used to id this */

        uint16_t lh_freelist;    /* chunk head of free list */
        uint8_t lh_flags;        /* ZLF_* flags */
        uint8_t lh_pad2[11];
    } l_hdr; /* 2 24-byte chunks */

    uint16_t l_hash[1];
} zap_leaf_phys_t;
\end{lstlisting}

Внутри каждой записи хранится еще одна хеш-таблица, для индексации в которой
используются также используются биты 64-битного хеша имени, но отличные от тех,
что использовались для индексации в таблице указателей. Для разрешение коллизий
на этом уровне так же используются цепочки.

\subsection{Интерфейс POSIX}

За предоставление интерфейса файловой системы поверх рассмотренного выше
интерфейса объектов отвечает ZFS POSIX Layer (ZPL). ZPL предоставляет наружу
интерфейс POSIX совместимой файловой системы.

ZPL хранит содержимое файловой системы в виде набора объектов. Каталоги
реализованы в виде рассмотренных ранее ZAP объектов, где имя является именем
файла или каталога внутри некоторой директории, а значение хранит номер объекта,
который соответствует файлу/каталогу.

Для хранения таких атрибутов как владелец, группа, время создания, обновления и
доступа, а также других атрибутов файлов/каталогов в Unix-подобных системах
используется так называемый дополнительный буфер (dn\_bonus) внутри
dnode\_phys\_t, т. е. вся специфичная для файловой системы информация хранится
прямо внутри структуры описывающей объект.

\section{Btrfs}

Btrfs - файловая система для ядра Linux, которая основана на использовании COW
модификации B+ деревьев~\cite{BCOW}. B+ деревья в Btrfs используются для всего:
\begin{itemize}
  \item для хранения информации о занятых/свободных участках диска используется
        дерево экстентов;
  \item для хранения информации о файлах/каталогах используется дерево файловой
        системы, которое хранит индексные узлы (inode) и записи каталогов
        (dentry);
  \item корни всех деревьев хранятся в еще одном дереве, которое называют
        корневым.
\end{itemize}

Как и ZFS, Btrfs обладает очень богатой функциональностью, обсуждение которой
выходит за рамки данной работы. Вместо этого в данной работе поверхностно
обсуждается основная индексная структура лежащая в основе Btrfs.

\subsection{B деревья и их вариации}

B деревья - это сильноветвистые идеально сбалансированные деревья поиска.
Классические алгоритмы вставки и удаления ключей из B дерева описанные в
~\cite{BTree} состоят из двух проходов по дереву:
\begin{itemize}
  \item проход от корня вниз по дереву для поиска позиции для вставки или ключа
        для удаления;
  \item балансировка начиная от измененного узла вверх в сторону корня.
\end{itemize}

Для балансировки используются несколько операций:
\begin{itemize}
  \item если после вставки в узел количество ключей в узле стало слишком
        большим, то узел расщепляется на два, а родительский узел добавляется
        ссылка на новый узел;
  \item если после удаления из узла количество ключей в узле стало слишком
        маленьким, то можно либо перераспределить ключи соседних узлов либо
        объединить узел с одним из соседей, при этом опять же придется изменять
        родительский узел.
\end{itemize}

B+ деревья поиска - это вариации B деревьев, в которой значения всегда хранятся
в листьях. Внутренние узлы хранят только ключи необходимые для поиска нужного
листа.

Для B+ деревьев на протяжении времени было предложено несколько схем
синхронизации, который позволяют выполнять конкурентные запросы к индексу
сравнительно эффективно. Одним из лучших таких подходов считается модификация,
которая называется B-link деревья предложенная в~\cite{BLink}. Однако в данной
модификации узлы одного уровня связаны в список и эта связь является
существенной для работы алгоритма. К несчастью ссылка на соседний узел делает
этот подход непригодным для COW.

\subsection{COW B+ деревья}

COW B+ деревья предложенные в~\cite{BCOW} отличаются от большинства классических
подходов в двух аспектах:
\begin{itemize}
   \item отсутствует ссылка на соседние узлы, только ссылки из родителя на
         ребенка;
   \item алгоритмы вставки и удаления выполняют проактивную балансировку по
         мере прохода по дереву от корня к листу.
\end{itemize}

Суть проактивного подхода к балансировке заключается в следующем. Если при
выполнении операции в ставки ключа в дерево на пути от корня у нужному листу
был найден полный узел дерева, т. е. узел вставка в который приведет к его
расщеплению, то это расщепление выполняется сразу же не дожидаясь, пока вставка
в этот узел действительно произойдет. Аналогичным образом происходит и при
операции удаления, только в этом случае алгоритм объединяет узлы с малым числом
ключей с соседними узлами или перераспределяет ключи.

По мере прохода по дереву от корня к листу для синхронизации используется подход
называемый lock coupling или crabbing. Суть подхода заключается в том, что
блокировка родительского узла освобождается только после того, как была
захвачена блокировка ребенка. При условии что проход по дереву всегда идет в
одном направлении возможность взаимной блокировки будет исключена. Однако если
работа с деревом всегда начинается с корневого узла, все потоки исполнения
должны захватывать блокировку корня, что плохо способствует параллельной работе
с таким деревом. И если при использовании read/write блокировки читатели могут
работать параллельно, то потокам писателям придется захватывать блокировки в
эксклюзивное пользование.

